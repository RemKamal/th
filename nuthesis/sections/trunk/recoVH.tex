Only dilepton pairs having net charge of zero are considered as \ZtoLL~ candidates. Pairs of the prompt isolated leptons have to have a dilepton mass greater than 76 GeV. This ensures the orthogonality with HIG-17-006 bbVV analysis as well as helps selecting decays of real Z bosons.

Higgs boson decays are reconstructed from the b jet pairs utilizing only the two with the highest CMVAv2 discriminant value. 

Double Higgs object is computed as a sum of Lorentz vectors of the \ZtoLL~ candidate, MET, and a \HBB~ candidate. Then, we compute the transverse mass of that object. Transverse mass definition that we follow is one of the commonly used and is logical in the sense that we subtract the longitudinal momentum component which leaves us with the transverse momentum components only (while the energy remains the total energy).

More precisely, as the z-component of the neutrinos' momentum is unknown, we form a pseudo transverse mass:

%$M_T...with-tilda = $ (further referred as transverse mass for brevity), where E and pz are the energy and momentum of the candidate defined above.''                                                               

$\tilde{M}_T(HH) = \sqrt{E^2 - p_{z}^2}$ (further referred as transverse mass for brevity), where $E$ and \
$p_z$ are the energy and the Z-axis component of the Lorentz energy-momentum vector of the HH candidate.

The resulting distribution is what will be used in the binned shape analysis with the Higgs Combination Tool following the section "Binned shape analysis" at the twiki:
\begin{center}
    \small{\texttt{https://twiki.cern.ch/twiki/bin/view/CMS/SWGuideHiggsAnalysisCombinedLimit}}
\end{center}
