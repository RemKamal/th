The bbZZ analysis uses the standard set of the CMS
reconstructed physics objects. We describe reconstruction of
electrons, muons, jets and b jets, and MET separately below:



\subsection{Electrons\label{sec:electrons}}
The Gaussian Sum Filter algorithm (GSF
Electrons)~\cite{Khachatryan:2015hwa} is used to reconstruct
electrons. The measurement selects electrons, which pass the following selection:
%loose set of preselection cuts in applied,
%namely, 
leading electron $\pt>25\GeV$ and subleading electron $\pt>15\GeV$, $|\eta|<2.5$, 
%$d_{xy}<0.05\cm$, $d_z<0.2\cm$ and 
an isolation cut of 0.06, for which the cone
of $0.3$ is used to compute the $\rho$-subtracted PF
isolation.  Lepton isolation is calculated as a scalar sum of
the transverse momentum $(p_{T})$ of all the charged and
neutral hadrons as well as photons around the lepton
(excluding the cone) divided by the $p_{T}$ of the lepton
itself. 
%POG recommended MVA ID WP Loose is further applied to the set of selected electrons.
%The cut on isolation in the measurement for electrons is $0.06$.

        
        %%To emulate the conditions of the HLT trigger, a set of offline cuts is applied further.

%% \texttt{pt>15 \& (}\\
%% \texttt{(abs(superCluster().eta)<1.4442 \& full5x5\_sigmaIetaIeta<0.012 \& }\\
%% \texttt{hcalOverEcal<0.09 \&}\\
%% \texttt{(ecalPFClusterIso/pt)<0.4 \& (hcalPFClusterIso/pt)<0.25 \&}\\
%% \texttt{(dr03TkSumPt/pt)<0.18 \& abs(deltaEtaSuperClusterTrackAtVtx)<0.0095 \&}\\
%% \texttt{abs(deltaPhiSuperClusterTrackAtVtx)<0.065) || }\\
%% \texttt{(abs(superCluster().eta)>1.5660 \& full5x5\_sigmaIetaIeta<0.033 \&}\\
%% \texttt{hcalOverEcal<0.09 \&}\\
%% \texttt{(ecalPFClusterIso/pt)<0.45 \& (hcalPFClusterIso/pt)<0.28 \&}\\
%% \texttt{(dr03TkSumPt/pt)<0.18)}\\
%% \texttt{).}

        In addition, a specific POG recommended working point is applied, which is a multivariate analysis (MVA) based criteria for classification of signal/background electrons. For this analysis we use the loose working point (another name can be WP90), as described in ~\cite{vhbbAN}. ID and ISO, as well as HLT SFs are applied.
%        \small{\texttt{https://twiki.cern.ch/twiki/bin/viewauth/CMS/ \\MultivariateElectronIdentificationRun2}}
%\normalsize



\subsection{Muons\label{sec:muons}}
        In this analysis we are using global muons reconstructed using the information from the tracker and muon-chamber \cite{CMS-PAS-MUO-10-002,Chatrchyan:2012xi}. The selection of muons is:
leading muon $\pt>20\GeV$ and subleading muon $\pt>15\GeV$, $|\eta|<2.4$,
%re is a loose preselection of $\pt>5\GeV$, $|\eta|<2.4$, $d_{xy}<0.5\cm$, $d_z<1.0\cm$, as well as 
a relative
isolation cut of 0.15, with the cone of $0.4$ used to compute $\Delta\beta$-subtracted PF isolation.
Finally, a tighter selection - POG recommended WP Loose is applied.~\cite{MuonsRun2}. ID, ISO, HLT and traker SFs are applied.
%The offline cut on isolation for muons is $0.15$.
%% Loose muon:
%% \begin{itemize}
%%   \item  Particle-Flow Muon:\\
%% \texttt{isPFMuon()}
%%   \item  is Global or Tracker Muon:\\
%% \texttt{isGlobalMuon() || isTrackerMuon()}
%%  \end{itemize}







\subsection{Jets\label{sec:jets}}
    Particle flow algorithm is used to reconstruct jets\cite{CMS-PAS-PFT-09-001,CMS-PAS-PFT-10-001}, with the help of the  $\text{anti}-k_T$ clustering algorithm having a distance parameter of $R=0.4$~\cite{Cacciari:2005hq,Cacciari:2008gp}.
    Reconstructed jets are further corrected for detector effects using specific correction determined from the data and MC. Only jets passing $|\eta|<2.4$ and  $(\pt > 30\GeV)$ are considered for the analysis. 
    All the necessary jet energy resolution (JER) and jet energy scale (JES) corrections provided by the JetMET group are applied ~\cite{JetMETgroup}.

%following this twiki:
%\begin{center}
 %   \texttt{https://twiki.cern.ch/twiki/bin/viewauth/CMS/JetResolution}    
%\end{center}






\subsection{Identification of b jets\label{sec:bjets}}
MVA technique combining the information about the impact parameter, identified secondary vertices, as well as soft lepton (if any) contained inside of the jet is used by the CMVA algorithm to identify b quark originated jets. The output is a continuous MVA discriminant ranging in value from -1 to +1. Optimal cut is determined by the POG for several working points. We use CMVAv2 medium working point  $(>0.4432)$. We checked all three WPs and WP Medium gives the best limits. b tag and mistag corrections are applied.

%\clearpage


\subsection{Missing transverse energy}\label{sec:MET}

MET type-1 corrected is calculated as an absolute value of the negative vector sum of all the visible PF candidates in the event. Then all the necessary corrections recommended by the POG are applied ~\cite{MissingETRun2Corrections} and on top, a set of filters related to the instrumental effects is employed ~\cite{MissingETOptionalFiltersRun2}. 
%following the twiki:
%\begin{center}
%\texttt{https://twiki.cern.ch/twiki/bin/viewauth/CMS/MissingETRun2Corrections}
%\end{center}
%On top, a set of filters related to the instrumental effects is employed:
%\begin{center}
%\texttt{https://twiki.cern.ch/twiki/bin/view/CMS/MissingETOptionalFiltersRun2}
%\end{center}
