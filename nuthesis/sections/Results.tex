%\section{Results}
%\label{sec:results}

The results in this measurement are obtained with the maximum likelihood fit. We perform a simultaneous fit of the SR and both CRs for both dielectron and dimuon channels using the likelihood function constructed as a product
of Poisson terms over all bins of the input \mTHH distributions in the three regions (SR, CRDY, CRTT) with Gaussian terms to constrain the nuisance parameters:

\begin{align*}
 L(r_{\text{signal}}, r_{k}|\text{data}) = \prod_{i=1}^{N_{\mathrm{bins}}}\frac{\mu_{i}^{n_{i}}\cdot e^{-\mu_{i}}}{n_{i}!}
\cdot \prod_{j=1}^{N_{\mathrm{nuisances}}} e^{-\frac{1}{2}\theta_{j}^{2}}
\end{align*}

where the product index $i$ refers to the bin of the input distributions, the product index $j$
refers to uncertainties accounted for by the fit model, and $n_i$ is the number of observed data
events in the bin $i$. The mean value for each of the Poisson distributions is computed as:


\begin{align*}
\mu_{i} &= r_{\text{signal}} \cdot S_{i} + \sum_{k}r_{k}\cdot B_{k,i},
\end{align*}


where $k$ refers to the background process $k$, and $B_{k,i}$ is the content of the bin $i$ of the background
shape for a process $k$, while $S_i$ is the content of the bin $i$ of the signal shape. The parameter $r_k$
sets the normalization of the background process $k$ while $r_{signal}$ is the signal strength parameter, all parameters $r$ are left free to float in the fit.
Two values of the signal strength parameter are of special interest:  $r_{signal} = 0$ describes the
background-only hypothesis, while $r_{signal} = 1$ corresponds to the case when the HH cross section
matches the cross section used for the initial signal normalization inspired by BSM models, 2pb in our case. 
The terms $\theta_j$ represent the set of nuisance parameters that are introduced into the likelihood
function as Gaussian constraints. 


Figure~\ref{fig:MCcomparisons}(~\ref{fig:MCcomparisons_radion}) shows the HH transverse mass distributions
for the signal and two control regions for both channels for the graviton (radion) resonance mass hypothesis with normalizations and shapes of all
components adjusted according to the best-fit values. The signal
sample is normalized to the cross section of 2~pb, a typical value for
predictions of WED models (e.g., at 300 GeV), and is further scaled, as indicated on the
Figure, to make it clearly visible. %The distributions exhibit a good agreement between data and the sum of the backgrounds.                                                                                                                                                                                                                                                                                                                                                                                                                                                                                                                                                                                                                                                                                                                                                                                




With the given 2016 dataset, the fit results show no evidence for HH production through a narrow
resonance, whose width is negligible in comparison to experimental
resolution, in the mass range from 250~GeV to 1~TeV. Thus, upper 95 \% confidence level limits on the
HH production cross section are set using the modified
frequentist CL$_s$ approach (asymptotic CL$_s$)~\cite{Junk:1999kv,LEP-CLs, HIG-11-011, Cowan:2010js}.

The observed and expected 95\% upper CL limits for the full mass range
and both resonances are listed in Table~\ref{tab:finalLimits}. We produce standard CMS Brazilian-flag type plots for the limits, we are shown in Fig.~\ref{fig:HHlimits}. The green and yellow
bands correspond to one and two standard deviations around
the expected limit, respectively. Since 450 GeV is the separation boundary between two mass regions: low mass and high mass, the limit calculation is performed with both of the BDTs at 450 GeV, where the discontinuity is
seen in the figure. The Figure also shows the expected production
cross section for a RS1 KK graviton/RS1 radion in WED models. %For a scenario with the curvature parameter $k/\overline{M}_{Pl}=0.1$ and the size of the extra dimension $kL=35$.                                                                                                                                                                                                                                                                                                                                                                                                                                                                                                                                                                                                                                                                                                                           
This cross section is computed in \cite{Oliveira:2014kla}
under the assumption of no mixing with the SM Higgs boson.











