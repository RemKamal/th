\hyphenation{ma-te-rials}
%%%%%%%%%%%%%%%%%%%%% Introduction %%%%%%%%%%%%%%%%%
\chapter{Introduction}
\label{ch:intro}

When one of the brightest physicists that has ever lived on Earth, Richard Feynman, was asked to summarise in one single phrase everything that we know about the world around us, he said, "all things are made of atoms" \cite{Feynman_atoms}. Feynman himself was the father of the quantum electrodynamics and definitely knew how to answer that question. In this simple response to general public he decided not go into quantum mechanics principles and rather illustrated at the highly abstract level that everything is made of smaller particles. For example, nowadays we know that atoms have nucleus and electrons on the electron shells, "orbiting" around the nucleus. The nucleus is positively charged proportionally to the number of protons it contains. To provide the stability of the nucleus of the heavy atoms our world also needs neutrons, but they have no electric charge. Also neutrons decay, a free neutron decays weakly in about 15 minutes, while the proton is stable, as far as we know from the experiments, and only in some Beyond the Standard Model (BSM) theories it is hypothesised that the proton lifetime is finite. However, the estimate of the proton lifetime is a value that is larger than the age of our Universe and is at least $10^{30}$ years \cite{DIMOPOULOS1982133}. Going further to a even smaller scale, we can say that protons and neutrons are not elementary, instead they are composed of point-like constituents that are called quarks. Quarks were proposed by Gell-Mann and also by Zweig to explain why Eightfold Way works  \cite{griffiths_hep}. Quarks come come in three families, or generations, and are arranged into doublets. Each doublet has an "up" quark with the charge $-1/3$ and a "down" quark with the charge $+2/3$. For antiquarks signs are reversed of course. These peculiar detail about their charge number being fractional was so much revolutionary at that time, that Gell-Mann decided not to publish his article in highly prestigious journals, but expecting a rejection, decided to go with the second tier one \cite{griffiths_hep}. In addition to a charge number, quarks also have "flavor" number assigned to them. For instance, a charm quark has $+1$ units of "charmness", while strange quark has $-1$ unit of "strangeness". All the other quark flavor field are zero for them. And this pattern is applied to all the other four quarks with the corresponding "quarkness". The last detail worth mentioning about quarks is that the mass of quarks increases from the first to the third family. No explanation exist in the SM, masses are the parameters in this theory, but perhaps the so-called "The Theory of Everything", which is to be written and had been a lifelong journey of another genius, Einstein \cite{aps_einstein}, will be able to explain such phenomenon. But it is not the whole description of quark properties. Another important characteristics of quarks has been revealed at the $e+e-$ colliders when analysers compared final cross sections in hadronic and muonic states. The theory was off by a factor of three. And this is how a quark color of three different types has been introduced: green, blue, and red. 



Since it was mentioned that the atoms contain electrons, it is a good time to talk about leptons. An electron, the first lepton to be officially observed in the experiment, was discovered by Thompson \cite{Davis:1989898} in 1897 when he was studying the properties of a cathode ray. That year is of a great importance, since it gave birth to an era of a modern particle physics. Almost 40 years later, in 1936, a muon was discovered \cite{Piccioni1996} in an experiment of Carl Anderson and Seth Neddermeyer who studied cosmic radiation. A muon is almost a copy of an electron, but is 207 times heavier. Analogously to quark families, leptons are also arranged in generations. Each generation is doublet that consists of a charged lepton (electron, muon or tau) with the charge $-1$ and a neutral lepton (corresponding electron, muon, or a tau neutrino). An electron and a muon neutrinos have been discovered in 1956 and 1962 respectively. The existence of the first one was deduced from the violation of conservation of energy in beta decay, while the muon neutrino \cite{PhysRevLett.9.36} was discovered by Schwartz, Lederman, and Steinberger during a several months long experiment with the pion beam, which was a counting experiment of events that pass the steel wall followed by an aluminum spark chamber. They accumulated 51 events of interest. Those events could not be due to electron neutrinos, since they will interact with the aluminum. The presence of muons in each event was a clear indication that those neutrinos were of a different kind, they were muon neutrinos. After some time, a tau and a tau neutrino were discovered in 1975 and 2000 correspondingly \cite{PhysRevLett.35.1489, Kodama:2000mp}. With that, the SM pattern for leptons was completed and long-awaited tau neutrino, which was theoretically speculated to exist, was finally observed experimentally. And likewise to families of quarks, lepton masses grow with the generation with the top quark from the 3rd generation being the heaviest particle in the whole SM. To help identify leptons of different families lepton numbers have been reserved: 1 unit of electron number to an electron and an electron neutrino, 1 unit of muon number to a muon and an muon neutrino, and 1 unit of tau number to a tau and a tau neutrino. And a final remark on neutrinos, in the SM they are massless, however, it has been show that they have a non-zero mass \cite{Bilenky:2014ema}. This fact is one of the main motivations for theorists to look for extensions of the SM. 

At the fundamental level the world is made of quarks and leptons. However, there must be rules that explain how quarks/leptons interact. These rules are referred to as fundamental forces of nature. As of now, there exist four: gravitational, weak, electromagnetic, and strong forces. The first one was governs the Universe at the macroscopic level: planets, solar systems, etc. The first theory of gravity was formulated by Newton \cite{Chandrasekhar:1187874} and then further developed by Einstein. A good historical perspective can be shown at \cite{Gutfreund:1980674}. Worth noting, that the gravitational force is not included in the SM. Attempts are ongoing to expand the SM, e.g., adding the graviton as a mediator, but no real success so far to have a renormalizable theory that would include both SM and gravity \cite{butterworth2014smashing}. 



Throughout this thesis the so-called natural system of units is adopted, in which $\hbar = c = 1$. This simplifies how many equations look and also makes a fine-structure constant $\alpha \approx 1/137$ dimensionless.
Following this convention \cite{Cottingham:1026625}, masses, momentum, and energies are measured in electronvolts (eV), with GeV ($10^9 eV$) and TeV ($10^12 eV$) being the most popular units in a modern high-energy physics due to energy regimes involved.





%=======================================

This thesis explores three components of elementary particle physics. The theoretical component: the standard model (SM) of particle physics gathers the best understanding of nature that is consistent with the experimental data and although it is extremely successful, it is known that SM is not the final version of a theory of everything. The data analysis component: statistical methods have been developed in order to obtain the most from that experimental data. The instrumentation component: detection systems are under continuous research and development in order to extend their capabilities and sensitivity and improve their precision. 

The context of SM is presented in Chapter \ref{ch:theory}, starting with a description of the basic components of the matter, quarks and leptons, and how they interact to produce the universe as it is. The language used in this description is the quantum field theory based on the principles of the gauge invariance, which states that the function used in describing a system is invariant under certain transformations; from the physics point of view, that gauge invariance means that a physical system can be described by more than one mathematical model. Although the choice of the gauge could make, for instance, the mathematical treatment of the model more or less challenging, it does not have any effect on the observables of the physical system, \ie, a physical system is independent of the model used to describe it. Along the document natural units will be used with $c=\hbar=1$. 

Interactions in the SM are represented in terms of the exchange of particles, known as gauge bosons. For instance, the electromagnetic interaction between two electrically charged particles is modeled as the exchange of a photon, while the strong interaction between quarks is modeled as the exchange of gluons; hence, the photon and gluon are two of the gauge bosons. In addition, there is an interaction that explains the mass of the elementary particles; this is the interaction with the so-called Higgs field.    

In the SM, the Higgs boson is responsible for providing the mass to the elementary particles, and a fundamental part of characterization of the Higgs boson consists of finding the way it interacts with the rest of elementary particles, \ie, how the Higgs boson couples with other particles. In this thesis the coupling of the Higgs boson with the top quark is investigated; in particular, the search for the production of a Higgs boson in association with a single top quark (\tH) is considered; the focus is on the $H \to WW$ , $H \to \tau\tau$, and $H \to ZZ$ decay modes that provide leptonic signatures in the final state. This process is of special interest due to its sensitivity to the relative sign of the top-Higgs coupling and the vector bosons-Higgs coupling; in addition, \tH process is sensitive to charge-parity (CP) symmetry violation effects related with the Higgs boson. Thus, a description of the incorporation of the Higgs boson in the SM and the specifics of the \tH process are also presented in Chapter \ref{ch:theory}.               
The SM is a very successful theory, capable of explaining and making predictions about a vast number of natural phenomena, therefore, it is under constant testing looking for evidences that verifies its predictions or that reveals the existence of physics beyond it and highlight the road to this new physics. Currently, experiments held at CERN\footnote{CERN stand for Conseil Europ\'een pour la Recherche Nucl\'eaire} provide data from proton-proton collisions used to explore the SM. The source of the data used in this thesis is the Compact Muon Solenoid experiment (CMS) for which a description is presented in Chapter \ref{ch:cms}.

Thanks to increasing development in computing, tools like Monte Carlo (MC) generators, simulation and reconstruction algorithms and software allow for evaluating the theory predictions and comparing them with real data. MC generators are used to create a set of simulated data samples that reflect the theoretical principles and details of the process under investigation, thus, predictions are obtained from the numerical solution of the mathematical models; however, a direct comparison with the data obtained from the experiments is not possible because of a variety of factors, for instance, the presence of the detection systems. The effect of the detection systems can be simulated and attached to the MC data samples such that the resulting samples account for these effects.

Experimental data are also processed; given that the whole detector is composed of several subdetectors, the information coming from these subdetection systems is combined to reconstruct the features of the particles produced after the proton-proton collision. The process of matching the information from different subdetection systems is known as event reconstruction. The result of the event reconstruction is a set of objects that are identified with the particles expected in the final state and that are predicted by the theory; in the \tH process case, those final state particles are leptons and jets. Chapter \ref{ch:gensimreco} presents the details about the computational tools used in this thesis.  

The statistical tools used to treat the data samples are described in Chapter \ref{ch:stat}; these tools include the Boosted Decision Trees (BDT) method employed to discriminate signal and background events based on their features, and the statistical inference methods used to account for the uncertainties introduced in the analysis and to extract the upper limits on the \tH+\ttH\ production cross section. 

%The core of this thesis is presented in Chapters \ref{ch:analysis} and \ref{ch:pixel}.

In Chapter \ref{ch:analysis}, the search for the production of a Higgs boson in association with a single top quark (\tH) is presented. First, the features of the signal and background processes are described; then, the MC and data samples considered, and the strategies oriented to identify the physics objects are defined. The event selection proceeds in two steps; first, an event pre-selection based on the signal features is performed; later, the signal is extracted based on BDT discriminators. As a result, an upper limit on the \tH+\ttH\ production cross section is set. Finally, the  sensitivity to CP-mixing in \tH process is investigated and upper limits on the \tH+\ttH\ production cross section are set.   

In Chapter \ref{ch:pixel}, the upgrade of the CMS forward pixel detection system (FPix) is presented. The HEP group at University of Nebraska - Lincoln (UNL) played a leading role in the so-called Phase 1 FPix upgrade, serving as a FPix modules assembly site; the assembly process was designed as a production line composed of several stages among which the gluing and encapsulation stages are described in detail. These stages were implemented using a semi-automated pick-and-place robotic system integrating vision, vacuum, and dispensing subsystems. The employment of the semi-automated setup, capable of providing a precision in location of about 10 $\mu$m, provides uniformity and speed up the module production. The commissioning of the assembly site started from scratch in late 2012 and by mid 2015 the production yields reached the same level as other experienced assembly sites.

Chapter \ref{ch:Conclusions} presents the conclusions from both analysis and hardware development sides.

