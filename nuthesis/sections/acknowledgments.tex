%%%%%%%% Start
I love reading acknowledgment chapters in other dissertations, and I have a number of incredible people to thank. Here is my story about the road to the Physics Ph.D. 

%%%%%%%%%%%% School
It all started in elementary school when my mother and grandfather taught me to solve math and physics problems. It was the definition of pure fun. I was never forced or asked to do my homework, nor did I need to work hard. Throughout the years I participated in a number of physics, math, brain teasers, and other ``hard-science'' Olympiads. I did it just because I liked it! I was also quite busy with history, economics, geography, and literature Olympiads. However, up until the last grade in high school, I had never gone to a high level, such as Nationals or close to it.

%%%%%%%%%%%% More at School
The situation changed in the twelfth grade when I won the bronze medal at the Physics Olympiad in the Kharkovska oblast (about 3 million people) and was invited to the eliminator, which determined who would represent the Kharkovska oblast at Ukraine's Nationals. This was enough to get admitted to the V. N. Karazin Kharkov National University.  Because of this while everyone was stressed about upcoming exams, I could enjoy the final year of school. I want to thank my high-school physics teacher Kurochka Violetta Mihailovna, who always thought that I was special and deserve it. 

%%%%%%%%%%%% At Karazin University
During my master's studies in Karazin University, learning from one of the best physics professors - Tonkopriad Alla Grigorievna - I heard that our physics alumni were doing Ph.Ds in the USA. She suggested contacting her previous student Dmitriy Hvostenko. Dima kindly explained everything from A to Z on how to get accepted to a Ph.D program in the USA and advised me to work closely with ``The Opportunity Program'' (EducationUSA), funded by the U.S. State Department. With my friend-brother and classmate Vladyslav Litichevskyi, we decided to pass TOEFL and GRE exams and continue our scientific careers in the USA. In Kharkov, the local EducationUSA Adviser Bulgakova Nataliya Borisovna kindly guided us through all aspects of the challenging path that we had chosen, providing us with the funds to take English lessons to prepare for exams and to cover related expenses. Thank you very much, Nataliya and EducationUSA! I am also thankful to my language tutor Olga Vorobieva for improving the level of my English.

%%%%%%%%%%% About Ph.D
I got accepted to several US Universities but did not know where to go. My friend Andrey Loginov, who unfortunately passed away in 2016 while still young, was a well-established Yale postdoc at CERN. He told me that at the University of Nebraska-Lincoln (UNL) he knew Ilya Kravchenko. Andrey added, ``Ilya is the real man.''. Since I have always valued relationships higher than any prestige or money, I decided not to go with other Universities and chose to go to Nebraska to work with Ilya instead. I have never regretted my choice.

    %%%%%%%%%%% Ph.D at UNL
I would like to thank Gregory Snow, who was a HEP group leader at UNL. He was the first person I met in the USA, and he helped me during my first few days there. At the beginning of my Ph.D, my Kharkov friend-brothers - Igor Cheredenko and Pavlo Ivashyna - supported me a lot. I am also thankful to my father who believed I could do it. He always loved and supported me. 

%%%%%%%%%%% More about Ph.D
My Ph.D years were... extremely tough. Taking theory classes 2.5 years, I thought it would never end... However, God sent me great people to help me go through this difficult period. I would like to thank my best friends Ekaterina Avdeeva, Elena Mar�a Echeverrr�a Mora, and Jose Andres Monroy Montanez. Without useful discussions with you, your support and help, I would have never achieved the GPA I have. I would also like to thank my first officemates Sumit Beniwal and Uday Singh. Later, I shared the office with another great person - Lei Yu, who introduced me to the Leetcode site. Thanks! Katya's husband, Sergey Adveev, is another great friend! With Katya and Sergey, we often enjoyed nice dinners and sport activities and talked about the C++ code. Special thanks to my friend-brother Anas Yazidi. 

%%%%%%%%%%% Three special people
In Lincoln, I met two special to me people, my closest friends: my coach and Godfather Gennady Yashirin and my soulmate Ruzanna Gansvind. Without you two, my US life would not have been so rich and colorful! You are a part of my family! I also would like to thank Lidiya Volyanyk for her mother's attitude of care towards me!

%%%%%%%%%% Fermilab
I also would like to thank my Fermilab (HATS and CMSDAS) friends and colleagues. Special thanks to Gabriele Benelli. Fermilab gave me a chance to teach at a number of HATS and CMSDAS schools for graduate students and postdocs and share my experience with Electron reconstruction at the CMS, while still being a graduate student myself. This was an amazing experience!

%%%%%%%%%% Travelling and ISSO
I also would like to thank UNL's International Student and Scholar Specialists - Natalia Meyer and Stephen Mattos - for understanding the need of my CERN-UNL travels and always providing me with timely signed documents and pieces of advice!

%%%%%%%%%%% CERN time
I have been visiting CERN since 2012 and moved to CERN full-time in 2015. I would like to thank my first roommate - Candice You - for being an amazing friend all these years, for making me interested in finance, and for the green book (by Zhou). 

%%%%%%%%%%%% CERN Officemates
I would like to thank my CERN officemates: Benjamin Stieger, Rebeca Gonzalez Suarez, Alejandro Gomez Espinosa, Savvas Kyriacou, and Linda Finco. Another great person was Stefanos Leontsinis. Many thanks to Benjamin for sharing your wisdom and advice. And many thanks to my friend-brother Jose! 

%%%%%%%%%%% Banda at CERN
I would like to thank my CERN friends from the former USSR: Alexander Nazarenko, Iskander Ibragimov, Dmitry Ustyushkin, Dimitri Potapov. 

%%%%%%%%%%%% EGamma group
I also would like to thank the CMS $e/\gamma$~group at CERN for giving me the chance to develop the electron identification working points for the whole CMS community. Special thanks to Matteo Sani.% and Giovanni Zevi Della Porta.

%%%%%%%%%%% More about CERN activities
I would like to thank my friend-brother Muhammad Alhroob for introducing me to the CERN Visit Service, where I have been a guide at CERN facilities for audiences ranging from middle school students to science professors. I would like to thank Marc Tassera, Vanya Guerre, Caroline Leroy, and Dominique Bertola. I met thousands of visitors, and more than twenty times I was asked to be a part of selfies - what an honor!
 
%%%%%%%%%%% CERN Finance Club
At CERN, I have been deeply involved in the activities of the Finance Club: managing the club, inviting top finance professionals to share their experience, and optimizing the existing club's portfolio with minimization and Monte Carlo methods. Special thanks to Christian Laner for being a good friend and a cool interview  practice problem partner. Also, thanks to Joaquim Creus Prats.

%%%%%%%%%%% CERN Boxing Club
CERN Boxing Club has been my second home for all these years. I trained people when I wanted to pass on the knowledge, trained with others when I wanted to have fun, and trained myself when I wanted to push really hard. I would like to thank my friends Federico Stagni, Vittorio Bencini, Francesco Romeo, Cristovao Andre Dionisio Barreto, Andreas Maier, Sylvain Ravat, and others. Thanks to my friend-brothers Pavel Makhov, Ivan Shvetsov, Misha Borodin, Giorgos Alexandropoulos, and Jacopo Nardulli. With the Boxing Club, I was participating in a highly prestigious CERN Relay Race with more than a hundred teams competing each year. I had the privilege to be a team captain of six runners several times. Each time we won the trophy. It was hard, it was fun, it was amazing!

%%%%%%%%%%%% CERN Alumni
I would like to thank my friends in the CERN Office of Alumni Relations - Rachel Bray and Laure Esteveny. It was a pleasure to work with them on events with such a high organization level! They let me lead several alumni events uniting top finance professionals, who used to work at CERN, with current CERN employees. It was a great time; we always had more than a hundred participants. Special thanks to Spyridon Papadopoulos.%, Irwin Sheer, and Michael Botlo.

%%%%%%%%%%%% About analysis
This measurement would have never materialized if I had not talked to Lesya Shchutska in 2016 about what kind of interesting measurement one could do then. She introduced me to Michele De Gruttola, a hard critic, a good mentor, and a great colleague to discuss physics over the coffee chats. Michele and I share not only the date we were born but also the same personality. Very soon he became my close friend. Thank you so much Michele and Lesya for your support and useful pieces of physics advice at the start of this physics analysis!

%%%%%%%%%%%% Analysis group Conveners
And, of course, such a difficult project would never have been completed without useful suggestions from Higgs, Higgs to ZZ, and double Higgs conveners. I would like to thank Roberto Salerno and Paolo Meridiani, Meng Xiao, Giacomo Ortona, Olivier Bondu, and Luca Cadamuro. Special thanks to Roberto! Thank you, Meng, Olivier, and Luca for friendship and support. And thanks a lot to Maxime Gouzevitch and Alexandra Carvalho.

%%%%%%%%%%%% The other bbZZ team
%I am thankful to the other $bbZZ$ team with whom we are preparing a combination paper: Emanuela Barberis, Apichart Hortiangtham, and David Morse.

%%%%%%%%%%%% Analysis Review Committee
I would like to thank my Analysis Review Committee (ARC) for their careful examination of this physics analysis. I was honored to have some of the best HEP physicists on my ARC: Jacobo Konigsberg, Sunil Somalwar, and Cecile Caillol. After the analysis became public, and I was applying for postdoc positions, Jacobo and Sunil (with Roberto, Lesya, and Ilya) wrote me very good (probably!) recommendation letters since I got several postdoc offers from the top schools. Thanks a lot, you are the best! And special thanks to Jaco!

%%%%%%%%%%%% Thesis Committee
I would like to thank UNL professors who agreed to be members of my committee - Bradley Shadwick, Peter Dowben, David Swanson, and Kenneth Bloom. Special thanks to Ken for a careful reading of this dissertation and providing me with so many useful comments. I would also like to thank the whole HEP group for being a part of my family while I was away from home. Special thanks to the best Department Chair ever, Professor Daniel Claes, for the support and encouragement! And, of course, I want to thank the Physics Department Staff for always being so helpful - Catherine Haley, Theresa Sis, Marjorie Wolfe, Joyce McNeil, Ellen Cox, Cyndy Petersen, Elizabeth Farleigh, Patricia Fleek, and Jennifer Becic. Thanks to Professors Bloom, Claes, Kravchenko, and Snow for supporting me all these years and giving me the opportunity to stay at CERN.

%%%%%%%%%%%% UNL HEP theory 
Special thanks to my theoretical particle physics colleagues and friends at UNL, Andrei Angelescu and Peisi Huang. They always were happy to discuss WED physics with me and answer some difficult (for me!) questions. Each time I came to UNL, we had great lunches with Andrei, during which we practiced French. Merci, Andrei!

%%%%%%%%%%%% UNL friends
I would like to thank my UNL friends. Thanks to my friends at the department Shawn Langan and Shi Cao. Also, big thanks to my fellow HEP graduate students: Joaquin Siado Castaneda, Caleb Fangmeier, and Robert Tabb. 

%%%%%%%%%% Ilya (adviser)
A specific paragraph should be devoted to a unique person - my boss and supervisor, my biggest critic, my mentor and best friend, my tough sparring partner, and an amazing particle physics lecturer - Ilya Kravchenko. When I came to Ilya, I did not know any programming. In addition, I thought quarks were as real as string theory. That means I had no preparation for the HEP, zero, nothing! It should be relatively easy to develop a great student if you start with a good one. But that would be too easy for Ilya. Instead, he decided to work with me when I had nothing but persistence and motivation. I guess Ilya saw that I was happy to work hard (I guess the only benefit of a tough childhood), and that I had  a willingness to learn whatever I needed to compete with the best. With Ilya I went from level zero to multiple postdoc offers from the most prestigious universities and several quant position offers from the biggest asset managers. From the bottom of my heart, thank you, Ilya!

%%%%%%%%%% Motivation
Also, I would like to thank people who have helped me go through hard times even though they were not in my life in person. I enjoyed and loved online Mathematical Physics lectures by Professor Carl Bender at the Perimeter Institute. That was real hard science, but it was presented in such an easy and fun way, that you would never see how hard it actually is. That is why now I love asymptotics and approximations so much! To keep my motivation high, there were times when I was getting lots of inspiration from Eric Thomas, a.k.a ET The Hip Hop Preacher. Also, I learned a lot about how to remain humble from Jocko Willink.

%%%%%%%%%% Church
I would like to thank Father Richard Klodnicki (Nick) and Father James Dank. Your Saint John of Kronstadt Orthodox Church in Lincoln has been my church for years and still is!

%Fran�ais
Mon passage par le CERN a �t� une belle occasion d'apprendre le fran�ais. Au d�but, j'avais commenc� � l'apprendre par moi-m�me. Mais force a �t� de constater que la complexit� de la t�che n�cessitait l'aide d'experts. Sur les conseils de plusieurs amis qui pouvaient appr�cier mes progr�s, je me suis inscrit au ``CERN Women's Club'' o� tout le monde y parlait un fran�ais impeccable. Finalement, j'ai trouv� quelques bons professeurs pour prendre des cours particuliers. Je dois admettre que pour moi, le fran�ais est une des langues les plus compliqu�es que j'ai pratiqu�. Si j'avais su � quoi m'attendre, j'y aurais r�fl�chi � deux fois avant de prendre cette d�cision. Malgr� tout, quels que soient les obstacles, tous sont surmontables avec de la pers�v�rance. Je souhaite � tous mes amis fran�ais du CERN de rester en bonne sant� et je suis certain, qu'un jour, nous reparlerons fran�ais ensemble!


%%%%%%%%%% Family
Most importantly, I would like to thank two people I love the most, my dearest mother and grandmother. My grandma had incredibly difficult challenges during the last five years of her life. During this period she lost both legs and had heart attacks. But even during this time, she did not cry or need any motivation. Instead, she always managed to support me and never lost hope. Without their support, I would have never finished this journey. As they would say, ``it is tough, but that is what makes a man.''. 

%%%%%%%%%% God
Finally, nothing would happen without the love and support of God and the mother of Jesus Christ! There were moments when I was close to losing hope, but now I see how those were important parts of the whole puzzle. 




