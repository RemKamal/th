I would like to tell my story and share my journey to the Physics Ph.D. It all started in the elementary school when my mother and my grandfather were solving math and physics problems with me. It was a pure definition of fun, I was never forced or asked to do my homework, neither I worked hard. Throughout the years, I participated in a number of physics, math, brain teasers, and other ``hard-science'' Olympiads just because I liked it. I also was quite busy with history, economics, geography, and literature Olympiads. However, up until the last grade in the high school, I have never went to the national level. 

The situation changed in the twelfth grade, when I won the bronze medal at the Physics Olympiad in the Kharkivska' oblast' (about 3 million people) and was invited to the eliminator to define who will represent the Kharkivska' oblast at the Ukraine Nationals. This was enough to get admitted to the V. N. Karazin Kharkiv National University.  I want to thank my physics teacher Kurochka Violetta Mihailovna, who always thought that I am special and deserve it. 

During the masters studies in Karazin University, from one of the best physics professors - Tonkopriad Alla Grigorievna - I heard that our physics alumni are studying in the USA. She suggested to contact Dmitriy Hvostenko. Dima kindly explained everything from A to Z on how to get accepted to the Ph.D program in the USA, and, most importantly, he told me about the program ``The Opportunity Program'', funded by the U.S. Department of State (EducationUSA). With my bro and classmate Vladyslav Litichevskyi we decided to pass the TOEFL and GRE exams and continue our scientific careers in the USA. In Kharkiv, the local EducationUSA Adviser Bulgakova Nataliya Borisovna guided us through all aspects of the challenge that we accepted, providing us with the funds to take English lessons to prepare for exams and covering related expenses. I am thankful to my English teacher Olga Vorobieva for improving the level of my language. 

I got accepted to several US Universities, but did not know where to go. My friend Andrey Loginov, who unfortunately passed away in 2016 still being in his 30s, was a well-established Yale postdoc sat CERN. He told me that at the University of Nebraska-Lincoln (UNL) he knows Ilya Kravchenko, and Andrey added, ``Ilya is the real man''. I always valued relationships higher than any prestige, so I decided not to go with higher ranked Universities and chose to work with Ilya instead. Later, I have never regretted my choice...

I would like to thank Greg Snow who was a HEP group leader at the UNL and passed away in 2019. He was the first person whom I met in the USA. He met me in the Lincoln airport and kindly drove me around to help me buy food and other stuff for first days in the USA. My Ph.D years were ... real dog years. And I know what I talk about after spending 8 years in the graduate school. Taking theory classes first two and a half years, I thought it would never end. However, God sent me great people to help me go through this tough period. I would like to thank my best friends Ekaterina Avdeeva, Elena Mar�a Echeverrr�a Mora, and Jose Andres Monroy Montanez. Without your help, support, and helpful discussions, I would have never got the GPA I have. I also would like to thank my first officemates Sumit Beniwal and Uday Singh. Later, I shared the office with another great person - Lei Yu, who introduced me to the Leetcode site, thanks! Katya's husband, Sergey Adveev, is another great friend whom I want to thank for being a part of my Lincoln life. We enjoyed together nice dinners, sport activities, and talked about the $C++$ code. Finally, in Lincoln I met two other special to me people, my closest friends: my coach and Godfather Gennady Yashirin my soulmate Ruzanna Gansvind. Without you two, my US life would not be that rich and colorful! You are a part of my family!

Professor Ken Bloom for a careful reading of this dissertation and providing me with so many comments to improve the text

Profs ... for for accepting to be in my committee


Lidiya Volyanyk 

 
 



