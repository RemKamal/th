I love reading acknowledgement chapters in other dissertations and I have a number of people to thank, so here is my story about the road to the Physics Ph.D. 

It all started in the elementary school when my mother and my grandfather were solving math and physics problems with me. It was a pure definition of fun, I was never forced or asked to do my homework, neither I worked hard. Throughout the years, though, I participated in a number of physics, math, brain teasers, and other ``hard-science'' Olympiads. Just because I liked it! I also was quite busy with history, economics, geography, and literature Olympiads. However, up until the last grade in the high school, I have never went to the national level. 

The situation changed in the twelfth grade, when I won the bronze medal at the Physics Olympiad in the Kharkivska' oblast' (about 3 million people) and was invited to the eliminator to define who will represent the Kharkivska' oblast at the Ukraine Nationals. This was enough to get admitted to the V. N. Karazin Kharkiv National University.  While everyone was stressed about upcoming exams, I could enjoy the final year of the school. I want to thank my high-school physics teacher Kurochka Violetta Mihailovna, who always thought that I am special and deserve it. 

During the masters studies in Karazin University, from one of the best physics professors - Tonkopriad Alla Grigorievna - I heard that our physics alumni are studying in the USA. She suggested to contact Dmitriy Hvostenko. Dima kindly explained everything from A to Z on how to get accepted to the Ph.D program in the USA, and, most importantly, he told me about the program ``The Opportunity Program'' (EducationUSA), funded by the U.S. Department of State. With my bro and classmate Vladyslav Litichevskyi, we decided to pass TOEFL and GRE exams and continue our scientific careers in the USA. In Kharkiv, the local EducationUSA Adviser Bulgakova Nataliya Borisovna kindly guided us through all aspects of the challenging path that we accepted, providing us with the funds to take English lessons to prepare for exams and covering related expenses. I am also thankful to my language tutor Olga Vorobieva for improving the level of my English.

I got accepted to several US Universities, but did not know where to go. My friend Andrey Loginov, who unfortunately passed away in 2016 still being in his 30s, was a well-established Yale postdoc sat CERN. He told me that at the University of Nebraska-Lincoln (UNL) he knows Ilya Kravchenko. Andrey added, ``Ilya is the real man''. Since I always valued relationships higher than any prestige, I decided not to go with other higher ranked Universities and chose to go to Nebraska to work with Ilya instead. Later, I have never regretted my choice...

I would like to thank Greg Snow, who was a HEP group leader at the UNL. He was the first person I met in the USA and helped me during my first few days in the USA. At the beginning of my Ph.D, my Kharkov bros supported me - Igor Cheredenko and Pavlo Ivashyna. I am also thankful to my father who believed I can do it. 

My Ph.D years were ... real dog years. Taking theory classes first two and a half years, I thought it would never end. However, God sent me great people to help me go through this tough period. I would like to thank my best friends Ekaterina Avdeeva, Elena Mar�a Echeverrr�a Mora, and Jose Andres Monroy Montanez. Without useful discussions with you and your help, I would have never got the GPA I have. I also would like to thank my first officemates Sumit Beniwal and Uday Singh. Later, I shared the office with another great person - Lei Yu, who introduced me to the Leetcode site, thanks! Katya's husband, Sergey Adveev, is another great friend whom I want to thank for being a part of my Lincoln life. With Katya and Sergey, we enjoyed together nice dinners, sport activities, and talked about the $C++$ code. Finally, in Lincoln I met two other special to me people, my closest friends: my coach and Godfather Gennady Yashirin and my soulmate Ruzanna Gansvind. Without you two, my US life would not be that rich and colorful! You are a part of my family! I also would like to thank Lidiya Volyanyk for her mother's care attitude towards me!

I have been visiting CERN since 2012 and moved to CERN full-time in 2015. I would like to thank one of my first roommates Candice You for being an amazing friend, for making me interested in finance, and for the green book (by Zhou). 

I would like to thank Muhammad Alhroob 


At CERN I have been deeply involved into activities of the Finance Club. I would like to thank Christian Laner for being a good friend and an interview problem practice partner. Also thanks to Quim, Pascal, and Seem.

CERN Boxing Club has been my second home for all these years, I trained people when I wanted to pass on the knowledge, trained with others when I wanted to have fun, and trained myself when I wanted to push hard. I would like to thank my friends Fede, Vittorio, Francesco, Chris and others. Special thanks to Misha Borodin, Ivan Shvetsov, Giorgos Alexandropoulos, and Jacopo Nardulli. With Boxing Club we were participating in a highly prestigious CERN Relay Race. I had a privilege to be a captain of the team of six several times. Each time we won the trophey since we gave our best, thank you guys!




Alumni - Rachel and X , Spyros...

Professor Ken Bloom for a careful reading of this dissertation and providing me with so many comments to improve the text

Profs ... for for accepting to be in my committee



 
 



