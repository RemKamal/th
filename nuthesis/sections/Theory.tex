\hyphenation{se-para-tion}
\hyphenation{theo-re-ti-cal}
\hyphenation{handed-ness}
\hyphenation{fo-llo-wing}
\hyphenation{ac-cor-ding}

%______________________ Theory ______________________
\chapter{Theoretical approach}
\label{ch:theory}

%______________________ INTRODUCCION ______________________
\section{The Standard Model}
The SM uses the Lagrangian formalism to describe the interactions of elementary particles and fields. The SM Lagrangian can be split into four main terms \cite{Mozer:2016wzi}:

\begin{itemize}
\item Gause bosons and their interactions
\item Fermions and their interactions with the gauge bosons
\item Higgs boson, its self-interaction, and interaction with the gauge bosons to give them mass, which is not possible solely by the $\Lagr_{YM}$
\item Fermions and their interactions with the Higgs boson, which through the Yukawa mechanism gives mass to fermions
\end{itemize}
 
or equivalently:
\begin{equation}\label{lagr_SM}
\Lagr_{SM} = \Lagr_{YM} + \Lagr_{ferm} + \Lagr_{H} + \Lagr_{Yuk} 
\end{equation}

The first term in the SM Lagrangian:
\beqn\label{lagr_YM}
\Lagr_{YM} = 	-\frac{1}{4}W^i_{\mu\nu}(x)W_i^{\mu\nu}(x) -\frac{1}{4}B_{\mu\nu}(x)B^{\mu\nu}(x) -\frac{1}{4}G^a_{\mu\nu}(x)G_a^{\mu\nu}(x)
\eeqn

where
\begin{align}
B_{\mu\nu}(x)   \equiv & \partial_\mu B_\nu -  \partial_\nu B_\mu \label{B_tensor} \\ 
W^i_{\mu\nu}(x) \equiv & \partial_\mu W^i_\nu(x) - \partial_\nu W^i_\mu(x) - g\varepsilon^{ijk}W^j_\mu W^k_\nu \label{W_tensor}\\
G^a_{\mu\nu}(x) \equiv & \partial_\mu G^a_\nu(x) - \partial_\nu G^a_\mu(x) - g_s f^{abc}G^b_\mu G^c_\nu \label{G_tensor}\\
\end{align}
with $i,j,k = 1,2,3$ and $a,b,c = 1, ..., 8$. The fields have the following connections to their underlying symmetries: $B_{\mu\nu}$ corresponds to $U(1)$ symmetry of the weak hypercharge $Y_k$ and "B" term is simply a kinematic term while "W" and "G" terms describe interactions among the corresponding bosons, where $W^i_{\mu\nu}$ corresponds to $SU(2)_I$ symmetry of the weak isospin $I^i_{w}$, and $G^a_{\mu\nu}$ corresponds to $SU(3)_c$ symmetry of the QCD color charge. $g$ and $\varepsilon$ are $SU(2)$ coupling and structure constants, while $g_s$ and $f$ are coupling and structure constants for $SU(3)$.

The next term in the SM Lagrangian shows how fermions interact with the gauge bosons. Notice, that the mass terms are still absent:
\beqn\label{lagr_ferm}
\Lagr_{ferm}= i \bar{\Psi}_L \slashed{D} \Psi_L  + i \bar{\psi}_{l_{R}}  \slashed{D} \psi_{l_{R}} +
i \bar{\Psi}_Q \slashed{D} \Psi_Q  + i \bar{\psi}_{u_{R}}  \slashed{D} \psi_{u_{R}} +
 i \bar{\psi}_{d_{R}}  \slashed{D} \psi_{d_{R}}
\eeqn

Above $\Psi$ represents a doublet of a charged lepton and a corresponding neutral lepton within the same lepton family of $SU(2)_L$, a letter Q is reserved for a family of quarks, and $\psi_R$ describes a right-handed leptonic singlet.

Gauge bosons interactions are present due to the derivative term:
\begin{align}\label{cov_der2}
D_\mu = \partial_\mu + ig I_w^i W_\mu^i+ ig' Y_w B_\mu + ig_s T_c^a G_\mu^a\\ 
\end{align}

Physical fields in this notation are represented by a linear combination of W and B fields:
\begin{align}\label{neutral_fields}
A_\mu = &  B_\mu \cos\theta_W + W^3_\mu \sin\theta_W \\ 
Z_\mu = & -B_\mu \sin\theta_W + W^3_\mu \cos\theta_W \nonumber 
\end{align}
\noindent where $\theta_W$ is known as the \ti{Weinberg angle} \cite{Weinberg:799984}.

These two first terms of the SM Lagrangian is enough to have a theory of fermions and bosons, but they have no mass \cite{Wolf:2015kua}. As discussed before, to ensure that weak bosons are massive, we need a Higgs term. Higgs mechanism enters the SM Lagrangian through the corresponding Higgs Lagrangian term given by 
\beqn\label{lagr_higgs}
\Lagr_H=(D_\mu\Phi)^\dagger(D^\mu\Phi) - V(\Phi) , \qquad V(\Phi)= - \mu^2(\Phi^\dagger\Phi) + \frac{\lambda}{4}(\Phi^\dagger\Phi)^2
\eeqn

where
\beqn
\Phi = \binom{\phi^+}{\phi^0 = (v+H + i\chi)/ \sqrt{2}} \quad \text{and} \quad v = 2 \sqrt{\frac{\mu^2}{\lambda}}
\eeqn

Here $\mu$ and $\lambda$ are parameters of the Higgs potential. The discovery of the Higgs boson and the measurement of its mass studies the $\mu$ parameter, double Higgs boson non-resonant searches target $\lambda$ parameter to know more precisely what is the shape of Higgs potential. After the SSB, the value of the Higgs field vacuum expectation value can be expressed in term of $\mu$ and $\lambda$ and is usually denoted by $v$ \cite{MonroyMontanez:2639240}.

\begin{figure}[H]
\centering
\includegraphics[width=0.6\textwidth]{hp2d}
\caption[SSB Potential form]{Shape of the Higgs potential before and after SSB that is determined at the leading orders by $\mu$ and $\lambda$ parameters. }
\label{hp2d}
\end{figure}


I
After adding $\Lagr_H$ and rearranging term, bosons have masses given by:
\beqn
M_W = \frac{gv}{2}, \quad  M_Z = \frac{M_W}{\cos{\theta_W}}, \quad M_H = \sqrt{2\mu^2}
\eeqn
 
The final piece of the SM Lagrangian is the Yukawa term, which Lagrangian is given by:
\beqn\label{lagr_Yuk}
\Lagr_{Yuk}=  - i \bar{\Psi}_{L}  G_l  \psi_{l_{R}} \Phi
- i \bar{\Psi}_{Q}  G_u  \psi_{u_{R}} \tilde{\Phi}
- i \bar{\Psi}_{Q}  G_d \psi_{d_{R}} \Phi + h.c.
\eeqn
where $\tilde{\Phi} = i \sigma^2 \Phi^*$

The masses of fermions enter the equation through the $3 \times 3$ matrices G, which are not known from the theory and are the parameters of the SM. The mass of the fermion is proportional to the Yukawa coupling of the corresponding fermion to the Higgs boson, which has already been mentioned when we discussed the Fig. \ref{coupling_ff}.

\section{Beyond the Standard Model}

Several BSM theories \cite{Huang:2017nnw, Dolan:2012ac, Kanemura:2016tan} predict a resonant production of the double Higgs boson events through a heavy narrow width resonance ($\sim O(1-10)$ GeV), which could be spin 0 or spin 2 particle \cite{Sirunyan:2018iwt}. In this particular analysis data is compared with respect to predictions from the Warped Extra Dimensions theory (WED) \cite{Oliveira:2014kla}. WED theory addressing the hierarchy problem adds additional fifth dimension to the 4-dimensional (4D) space-time. In the framework that Randall and Sundrum (RS) \cite{Randall:1999ee} followed 4D space then is nothing but an EFT approximation, where the radion or graviton may exist as Kaluza-Klein (KK) \cite{Uzawa:1999pg} excitation modes at the TeV scale. Since LHC had provided us with no evidence of the SM particles interacting with the additional RS dimensions, it is postulated that they are confined to 3-brane, or a wall. At the same time, gravity, which is not in the SM, can propagate freely in the full higher-dimensional space, so-called bulk. If/when the bulk is compactified, it may produce KK modes of the gravitons. In this analysis RS model with parameter k of the order of Planck scale and $\bar{M}_{Pl}$, a reduced 4D $M_{Pl}$ which is a function of the 5D Planck scale M and a parameter k with $k<M$, are assumed to satisfy the constraint $0.01 \leq k / \bar{M}_{Pl} \leq 1$, because values outside of this range are not applicable/or complicate the theory \cite{Davoudiasl:1999jd}. Considered in this search graviton and radion are thus RS KK graviton and RS radion particles that emerge in RS scenario with a mass of KK state of the order of TeV. 

Lagrangian energy stress tensor over fields i \cite{Forger:2003ut}

$\lambda_W \sim O$(TeV)







