Since the discovery of the Higgs boson in 2012 by the A Toroidal LHC ApparatuS (ATLAS) and 	The Compact Muon Solenoid (CMS)%\cite{Della Negra1560}
, most of the quantum mechanical quantities that describe the long-awaited Higgs boson have been measured. Due to an impeccable work of the LHC, dozens of $fb^{-1}$ of data have been delivered to both experiments. Finally, it became possible for analyses that have a very low cross section to observe rare decay modes of the Higgs boson, such as, ttH and VHbb. The only untouched territory is a double Higgs boson production. Data will not help us much either at the HL-LHC, the process will remain unseen even in the most optimistic scenarios, so one has to rely solely on new reconstruction methods as well as new analysis techniques. This thesis is addressing both goals. I have been blessed by an opportunity to work in the CMS electron identification group, where we have developed new electron identification algorithms. The majority of this thesis, however, will be devoted to the second goal of HL-LHC. We establish the techniques for the first ever analysis at LHC that searches for the double Higgs production mediated by a heavy narrow-width resonance in the $b\bar{b}ZZ$ channel:  $X \to HH \to b\bar{b}ZZ^{*} \to b\bar{b} \ell\ell\nu \bar{\nu}$. The analysis searches for a resonant production of a Higgs boson pair in the range of masses of the resonant parent particle from 250 to 1000 $GeV$. Both spin scenarios of the resonance are considered: spin 0 (later called "graviton") and spin 2 (later called "radion"). In the absence of the evidence of the resonant double Higgs boson production from the previous searches, we set upper confidence limits. When combined with other search channels, this analysis will contribute to the discovery of the double Higgs production and we would be able to finally probe the Higgs boson self-coupling. 