Since the discovery of the Higgs boson in 2012 by the ATLAS and CMS experiments%\cite{Della Negra1560}
, most of the quantum mechanical properties that describe the long-awaited Higgs boson have been measured. Due to the outstanding work of the LHC, over a hundred of fb$^{-1}$ of proton collisions data have been delivered to both experiments. Finally, it became sensible for analyses teams to start working with a very low cross section processes involving the Higgs boson, e.g., a recent success in observing ttH and VHbb processes. One of the main remaining untouched topics is a double Higgs boson production. However, an additional hundred of fb$^{-1}$ per year from the HL-LHC will not necessarily help us much with the SM double Higgs physics, as the process may remain unseen even in the most optimistic scenarios. The solution is to work in parallel on new reconstruction and signal extraction methods as well as new analysis techniques to improve the sensitivity of measurements. This thesis is about both approaches: we have used the largest available dataset at the time the analysis has been performed and developed/used the most novel analysis methods. One such method is the new electron identification algorithm that we have developed in the CMS electron identification group, to which I have had a privilege to contribute during several years of my stay at CERN.

The majority of this thesis is devoted to techniques for the first search at the LHC for double Higgs boson production mediated by a heavy narrow-width resonance in the $b\bar{b}ZZ$ channel:  $X \to HH \to b\bar{b}ZZ^{*} \to b\bar{b} \ell\ell\nu \bar{\nu}$. The measurement searches for the resonant production of a Higgs boson pair in the range of masses of the resonant parent particle from 250 to 1000 GeV using 35.9 fb$^{-1}$ of data taken in 2016 at 13 TeV. Two spin scenarios of the resonance are considered: spin 0 and spin 2. In the absence of the evidence of the resonant double Higgs boson production from the previous searches, we proceed with setting the upper confidence limits. %When combined with other search channels, this analysis will contribute to the discovery of the double Higgs production and we would be able to finally probe the Higgs boson potential using its self-coupling. 